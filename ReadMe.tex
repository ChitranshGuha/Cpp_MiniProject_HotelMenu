\documentclass{article}
\usepackage{enumitem}

\begin{document}

\section{Food Ordering System}

The aim of the C++ program is to create a simple food ordering system. It allows users to interact with a menu of food items, make selections, place orders, and perform certain operations related to their orders. Here are the main features and functionalities of the program:

\begin{enumerate}[label=\arabic*.]
    \item \textbf{Menu Management:}
    \begin{itemize}
        \item The program defines a list of food items categorized into different categories such as "Burgers," "Pizza," "Pasta," "Sandwich," "Hot Coffee," "Cold Coffee," and "Ice Cream."
    \end{itemize}
    
    \item \textbf{Ordering System:}
    \begin{itemize}
        \item Users can view the menu items, which are displayed in their respective categories.
        \item Users can select items by entering the number associated with the item.
        \item Users can specify the quantity of each item they want to order.
        \item The program maintains a list of ordered items along with their quantities.
    \end{itemize}
    
    \item \textbf{Order Operations:}
    \begin{itemize}
        \item Users can review their current order, which includes the names of ordered items, their quantities, and the total cost.
        \item Users can remove items from their order.
        \item Users can finish and submit their order.
    \end{itemize}
    
    \item \textbf{Table Booking:}
    \begin{itemize}
        \item Users can specify the number of people for table booking, simulating a reservation system for a restaurant.
    \end{itemize}
    
    \item \textbf{Error Handling:}
    \begin{itemize}
        \item The program includes error handling to validate user input, ensuring that quantities and choices are valid.
    \end{itemize}
    
    \item \textbf{Interactive Interface:}
    \begin{itemize}
        \item The program runs in a loop, providing an interactive command-line interface for users to navigate the menu and place orders.
    \end{itemize}
\end{enumerate}

Overall, the program provides a basic simulation of a restaurant's food ordering system. Users can select items from the menu, specify quantities, review their orders, and even book a table if needed. This program could serve as a starting point for the development of a more sophisticated restaurant management system with additional features, such as payment processing and order management.

\end{document}
